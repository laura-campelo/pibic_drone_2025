\section{Introdução}
O ser humano sempre teve interesse em saber como era dominar os céus da forma que alguns animais o faziam 
naturalmente. Muitas engenhocas foram criadas com o intuito de atingir esse objetivo, sendo os drones uma das 
versões mais recentes dessas tentativas. Após vários outros sucessos, como o balão, o avião e o helicóptero, bem 
como certos fracassos como o dirigível, surgiu a necessidade de se ter veículos aéreos não tripulados (VANTs ou 
UAVs, em inglês) para a realização de tarefas perigosas, dispendiosas ou inviáveis de serem orquestradas por um 
meio de transporte tripulado \cite{ita}.

\subsection{Motivação}
Nessa linha, os veículos áereos não tripulados de quatro rotores, também chamados drones quadrirotores ou drones 
quadcópteros, tornaram-se bastante populares ao longo dos anos devido principalmente a sua capacidade de decolagem 
e pouso verticais (assim como as demais arquiteturas de drones), possuindo diversas aplicações atualmente. Entre 
elas, está o emprego na saúde pública \cite{drone_medico}, em que se destacam: o transporte de suprimentos médicos, 
tal como a entrega de sangue, vacinas, medicamentos essenciais (como soro antiofídico) e amostras laboratoriais de 
e/ou para áreas rurais em adição à redução do tempo de transporte de órgãos doados, minimizando danos e evitando 
congestionamentos terrestres; a assistência em desastres, tal como o transporte de alimentos, água, medicamentos e 
dispositivos médicos (como desfibriladores), além de auxiliar no resgate de vítimas em áreas de colapso ou 
naufrágios; e a vigilância e monitoramento em áreas de risco (como áreas onde há contaminação nuclear).

Outra aplicação proeminente é no mercado de entregas de produtos em geral \cite{drone_entrega} em que se observa 
grande vantagem no aumento da rapidez ocasionada pela capacidade de se evitar áreas congestionadas bem como de 
atender satisfatoriamente as crescentes demandas do \textit{e-commerce}, o que possibilita um ganho expressivo nos 
lucros. Por fim, é possível destacar também o uso desses equipamentos na agricultura \cite{drone_agricultura} para o 
monitoramento das lavouras, permitindo intervenções mais eficientes e precisas; para a pulverização de insumos como 
fertilizantes e defensivos agrícolas, otimizando o uso desses recursos; e para o controle de pragas e doenças por meio 
da utilização de sensores para capturar dados meteorológicos e identificar doenças precocemente.

\subsection{Estado da Arte}
Nos últimos anos, vários estudos têm investigado formas de controlar drones quadrirotores, tentando equilibrar estabilidade, precisão e robustez em diferentes cenários. Em \cite{pfeifer}, o foco foi no projeto e controle de quadrirotores voltados para monitoramento, vigilância e resgate. Combinou-se modelagem dinâmica usando equações de Newton-Euler, uma estrutura em ``X'' e sensores diversos. Os controladores lineares com alocação de polos e LQR, combinados com filtro de Kalman, foram simulados no MATLAB/Simulink e testados via HIL. Os resultados mostraram respostas suaves, menor consumo de energia e boa rejeição a ruídos, mas o HIL apresentou algumas instabilidades devido a vibrações e atrasos.

Já o trabalho de \cite{he} propôs um controle fuzzy adaptativo PID para quadrirotores usados na agricultura, tentando superar limitações do PID clássico. Eles modelaram o quadrirotor com equações de Newton-Euler simplificadas e fizeram o controlador ajustar dinamicamente os ganhos $K_p$, $K_i$ e $K_d$ usando lógica fuzzy, funções gaussianas e 49 regras. Nas simulações, o overshoot caiu $40\%$, o tempo de assentamento foi reduzido à metade, os distúrbios externos foram melhor rejeitados e a saída de controle ficou $25\%$ mais eficiente.

O estudo de \cite{1} tratou de um controle PD-PID para manter o drone em hover, aproveitando sensores MEMS/NEMS. O sistema combinava controladores de altitude, orientação e posição, com PID adaptativo que ajusta automaticamente os ganhos proporcional e derivativo, garantindo que não haja erros em estado estacionário. As simulações mostraram que o PD-PID estabiliza rapidamente o drone em altitudes entre 2 e 4 metros, com tempos de 2 a 25 segundos, ângulos limitados a $\pm20°$ e sem overshoot significativo, o que indica precisão e robustez. Os autores comentam que, em futuras implementações em hardware real, seria interessante testar a resistência a falhas de atuadores e a ventos.

A tese de \cite{elkholy} explorou tanto controladores lineares quanto não lineares, incluindo PID/PD com algoritmo genético, PD com gain scheduling, Sliding Mode Control (SMC) e Backstepping, para lidar com a dinâmica complexa do quadrirotor, incluindo efeitos aerodinâmicos e a dinâmica dos rotores. O PD se mostrou simples e eficiente, o gain scheduling melhorou o desempenho em trajetórias variáveis, o SMC ofereceu robustez a distúrbios com chattering reduzido e o Backstepping trouxe respostas suaves e baixo overshoot. Em geral, o estudo mostrou que controladores não lineares têm vantagem em regimes mais complexos e sob vento.

Por fim, a tese de \cite{sabatino} derivou o modelo do quadrirotor via Newton-Euler, projetou LQR e ainda desenvolveu dois controladores não lineares baseados em linearização por feedback: inversão dinâmica e linearização exata. As simulações mostraram que o LQR é mais lento e suave, a linearização exata mais rápida e a inversão dinâmica ainda mais veloz, embora com um pequeno aumento de overshoot, mas todos conseguiram seguir trajetórias complexas de forma consistente.

No geral, esses trabalhos mostram que, embora controladores lineares como LQR ainda sejam úteis por sua simplicidade, abordagens adaptativas e não lineares, como fuzzy PID, gain scheduling, SMC e Backstepping, são mais eficientes em cenários reais, com distúrbios e trajetórias complexas.

\subsection{Objetivos}
\lipsum[1]
