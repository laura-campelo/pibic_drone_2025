\section{Introdução}
O ser humano sempre teve interesse em saber como era dominar os céus da forma que alguns animais o faziam 
naturalmente. Muitas engenhocas foram criadas com o intuito de atingir esse objetivo, sendo os drones uma das 
versões mais recentes dessas tentativas. Após vários outros sucessos, como o balão, o avião e o helicóptero, bem 
como certos fracassos como o dirigível, surgiu a necessidade de se ter veículos aéreos não tripulados (VANTs ou 
UAVs, em inglês) para a realização de tarefas perigosas, dispendiosas ou inviáveis de serem orquestradas por um 
meio de transporte tripulado \cite{ita}.

\subsection{Motivação}
Nessa linha, os veículos áereos não tripulados de quatro rotores, também chamados drones quadrirotores ou drones 
quadcópteros, tornaram-se bastante populares ao longo dos anos devido principalmente a sua capacidade de decolagem 
e pouso verticais (assim como as demais arquiteturas de drones), possuindo diversas aplicações atualmente. Entre 
elas, está o emprego na saúde pública \cite{drone_medico}, em que se destacam: o transporte de suprimentos médicos, 
tal como a entrega de sangue, vacinas, medicamentos essenciais (como soro antiofídico) e amostras laboratoriais de 
e/ou para áreas rurais em adição à redução do tempo de transporte de órgãos doados, minimizando danos e evitando 
congestionamentos terrestres; a assistência em desastres, tal como o transporte de alimentos, água, medicamentos e 
dispositivos médicos (como desfibriladores), além de auxiliar no resgate de vítimas em áreas de colapso ou 
naufrágios; e a vigilância e monitoramento em áreas de risco (como áreas onde há contaminação nuclear).

Outra aplicação proeminente é no mercado de entregas de produtos em geral \cite{drone_entrega} em que se observa 
grande vantagem no aumento da rapidez ocasionada pela capacidade de se evitar áreas congestionadas bem como de 
atender satisfatoriamente as crescentes demandas do \textit{e-commerce}, o que possibilita um ganho expressivo nos 
lucros. Por fim, é possível destacar também o uso desses equipamentos na agricultura \cite{drone_agricultura} para o 
monitoramento das lavouras, permitindo intervenções mais eficientes e precisas; para a pulverização de insumos como 
fertilizantes e defensivos agrícolas, otimizando o uso desses recursos; e para o controle de pragas e doenças por meio 
da utilização de sensores para capturar dados meteorológicos e identificar doenças precocemente.

\subsection{Estado da Arte}
\lipsum[1]

\subsection{Objetivos}
\lipsum[1]