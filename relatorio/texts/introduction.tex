\section{Introdução}

O ser humano sempre teve interesse em saber como era dominar os céus da forma que alguns animais o faziam 
naturalmente. Muitas engenhocas foram criadas com o intuito de atingir esse objetivo, sendo os drones uma das 
versões mais recentes dessas tentativas. Após vários outros sucessos, como o balão, o avião e o helicóptero, bem 
como certos fracassos como o dirigível, surgiu a necessidade de se ter veículos aéreos não tripulados (VANTs ou 
UAVs, em inglês) para a realização de tarefas perigosas, dispendiosas ou inviáveis de serem orquestradas por um 
meio de transporte tripulado \cite{ita}.

Nessa linha, os veículos áereos não tripulados de quatro rotores, também chamados drones quadrirotores ou drones 
quadcópteros, tornaram-se bastante populares ao longo dos anos, possuindo diversas aplicações atualmente. Entre 
elas, está o emprego na saúde pública \cite{drone_medico}, em que se destacam: o transporte de suprimentos médicos, 
tal como a entrega de sangue, vacinas, medicamentos essenciais (como soro antiofídico) e amostras laboratoriais de 
e/ou para áreas rurais em adição à redução do tempo de transporte de órgãos doados, minimizando danos e evitando 
congestionamentos terrestres; a assistência em desastres, tal como o transporte de alimentos, água, medicamentos e 
dispositivos médicos (como desfibriladores), além de auxiliar no resgate de vítimas em áreas de colapso ou 
naufrágios; e a vigilância e monitoriamente em áreas de risco (como áreas onde há contaminação nuclear).