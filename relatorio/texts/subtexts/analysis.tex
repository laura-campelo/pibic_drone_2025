\section{Análise do Sistema}
Para construir controladores mais confiáveis e robustos, faz-se necessário entender como o sistema se comporta ao 
reagir a entradas variadas. Essa compreensão permite, então, analisar algumas propriedades desse sistema, como 
estabilidade, controlabilidade, observabilidade e, inclusive, o seu grau de acoplamento, informações essas 
que se mostram valiosas para a extração de intuições quanto à viabilidade de implantação dos controladores.

\subsection{Estabilidade}
Neste trabalho, abordar-se-á dois critérios para a análise de estabilidade: BIBO (\textit{Bounded Input, Bounded 
Output}) e Lyapunov. Para o primeiro critério, 
um sistema é considerado estável se reage de forma limitada dada uma entrada limitada. No caso do drone, por exemplo, 
se for mantida uma entrada constante e de mesmo valor nos quatro rotores, o drone continuaria subindo indefinidamente 
caso a entrada inserida fosse forte o bastanta para vencer a resistência aplicada pelo próprio peso do aparelho. Dessa 
forma, um drone é considerado um sistema BIBO instável.

Já para Lyapunov, que é um critério mais abrangente que o anterior, busca-se verificar a convergência das respostas 
do sistema em direção ao equilíbrio. Quando estamos tratando de sistemas lineares, uma forma de verificar essa 
convergência é por meio dos autovalores da matriz $\mathbf{A}$ (definida anteriormente) de modo que, caso eles possuam 
partes reais negativas, o sistema é considerado assintoticamente estável segundo Lyapunov. Assim, os autovalores 
da matriz dada são:
\begin{equation*}
    \left[\begin{array}{cccccccccccc}
        0 & 0 & -0.534 & -0.534 & 0 & 0 & 0 & 0 & -0.534 & 0 & 0 & 0
    \end{array}\right].
\end{equation*}

Observando os valores acima, nota-se que o sistema não pode ser considerado assintoticamente estável segundo Lyapunov 
pois nem todas as partes reais de seus autovalores são negativas, havendo vários deles nulos.

\subsection{Controlabilidade}
A controlabilidade está associada à capacidade de um sistema ser controlado de forma eficaz. Em outras palavras, ela 
analisa a capacidade das entradas disponíveis de levar o sistema, em um espaço de tempo finito, de qualquer um de seus 
estados, considerado como inicial, até qualquer outro estado, considerado como final.

Na prática, para sistemas lineares em formato de espaço de estados (ver seção anterior) essa análise é feita a 
partir da matriz de controlabilidade (mostrada abaixo), em que, para que o sistema 
seja considerado controlável, o posto dessa matriz deve ser igual à quantidade $n$ de estados do sistema (neste trabalho, 
$n=12$).
\begin{equation*}
    \mathbf{C_M} = \begin{bmatrix}
    B & AB & A^2B & \cdots & A^{n-1}B
    \end{bmatrix}
\end{equation*}

Realizando os devidos cálculos com as matrizes $\mathbf{A}$ e $\mathbf{B}$ dadas anteriormente, obtém-se um posto 
completo e, portanto, conclui-se que o sistema é controlável.

\subsection{Observabilidade}
A observabilidade está associada à capacidade de um sistema ser monitorado de forma eficaz. Em outras palavras, ela 
analisa a capacidade de se reconstruir os estados do sistema a partir das entradas e das saídas conhecidas.

Na prática, para sistemas lineares em formato de espaço de estados (ver seção anterior) essa análise é feita a 
partir da matriz de observabilidade (mostrada abaixo), em que, da mesma forma que para a controlabilidade, 
o posto dessa matriz deve ser igual à quantidade $n$ de estados do sistema (neste trabalho, 
$n=12$) para que ele seja considerado observável.
\begin{equation*}
    \mathbf{O_M} = \begin{bmatrix}
    C \\
    CA \\
    CA^2 \\
    \vdots \\
    CA^{n-1}
    \end{bmatrix}
\end{equation*}

Realizando os devidos cálculos com as matrizes $\mathbf{A}$ e $\mathbf{C}$ dadas anteriormente, obtém-se novamente 
um posto completo e, portanto, conclui-se que o sistema é observável.

\subsection{Acoplamento}
Para terminar esta análise, é preciso verificar o nível de acoplamento do sistema. Em outras palavras, 
faz-se necessário avaliar quanto cada entrada (ou seja, $U_1, U_2, U_3, U_4$ definidas na seção 
\ref{sec:ac_control}) influencia cada saída (por simplicidade, $z, \phi, \theta, \psi$ definididas na 
Tabela \ref{tab:vars}). Na prática, utiliza-se a Matriz de Ganhos Relativos (RGA) definida como:
\begin{equation*}
    \mathbf{RGA} = G \cdot {(G^{-1})}^T,
\end{equation*}

\noindent em que $G$ é a matriz de ganhos do sistema definida em estado estacionário. Aplicando a fórmula 
acima para o sistema em questão, obtemos:
\vspace{-0.4cm}
\begin{center}
    \begin{longtable}{|c|c|c|c|c|} % <-- note o uso de p{largura}
        \caption{Matriz de Ganho Relativo do Sistema} \vspace{-0.4cm}
        \label{tab:rga}
        \hline
        Variáveis & $U_1$ & $U_2$ & $U_3$ & $U_4$\\
        \hline
        \endfirsthead
        % \hline
        % \emph{Parâmetros} & \emph{Descrição} & \emph{Valores} & \emph{Unidade} \\
        % \hline
        \endhead
        \endfoot
        \caption*{\footnotesize{Fonte: Autora.}} \vspace{-2cm}
        \endlastfoot

        $z$ & $0.25$ & $0.25$ & $0.25$ & $0.25$\\ \hline
        $\phi$ & $0$ & $0.5$ & $0$ & $0.5$\\ \hline
        $\theta$ & $0.5$ & $0$ & $0.5$ & $0$\\ \hline
        $\psi$ & $0.25$ & $0.25$ & $0.25$ & $0.25$ \\
        \hline
    \end{longtable}
\end{center}
% \begin{table}[h!]
%     \centering
%     \caption{Matriz de Ganho Relativo do Sistema}
%     \begin{tabular}{c|cccc}
%         & $U_1$ & $U_2$ & $U_3$ & $U_4$\\
%         \hline
%         $z$ & $0.25$ & $0.25$ & $0.25$ & $0.25$\\
%         $\phi$ & $0$ & $0.5$ & $0$ & $0.5$\\
%         $\theta$ & $0.5$ & $0$ & $0.5$ & $0$\\
%         $\psi$ & $0.25$ & $0.25$ & $0.25$ & $0.25$
%     \end{tabular}%
%     \label{tab:rga}
% \end{table}

% \pagebreak

Para interpretar esses resultados, é preciso saber que valores iguais ou bastante próximos a $1$ 
querem dizer que aquele par de variáveis pode ser utilizado para um controle isolado do resto do sistema, 
enquanto valores diferentes indicam que as saídas são influenciadas pelas entradas em graus similares e, 
portanto, não podem ser facilmente separadas em pares de controle isolados.

Assim, observando a Tabela 
\ref{tab:rga}, nota-se que o drone é possui um grau de acoplamento bastante elevado e que, portanto, 
não é possível separá-lo em pares de variáveis controladas separadamente. Além disso, levando em conta 
apenas as entradas e saídas mostrada na Tabela \ref{tab:rga}, seriam necessários $12$ desacopladores para 
fazer uma separação melhor entre as influências mútuas, o que aumentaria desnecessariamente a complexidade 
do problema. Essa constatação levou a decisão de se lidar diretamente com o modelo não-linear e buscar 
alternativas de controle que se adaptassem a essa abordagem, já que opções mais clássicas seriam inadequadas 
devido à necessidade de projeto de muitos desacopladores.