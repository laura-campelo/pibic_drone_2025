\section{Análise do Sistema}
Para construir controladores mais confiáveis e robustos, faz-se necessário entender como o sistema se comporta ao 
reagir a entradas variadas. Essa compreensão permite, então, analisar algumas propriedades desse sistema, como 
estabilidade, controlabilidade, observabilidade e, inclusive, o seu grau de acoplamento, informações essas 
que se mostram valiosas para a extração de intuições quanto à viabilidade de implantação dos controladores.

\subsection{Estabilidade}
Neste trabalho, abordar-se-á dois critérios para a análise de estabilidade: BIBO (\textit{Bounded Input, Bounded 
Output}) e Lyapunov. Para o primeiro critério, 
um sistema é considerado estável se reage de forma limitada dada uma entrada limitada. No caso do drone, por exemplo, 
se for mantida uma entrada constante e de mesmo valor nos quatro rotores, o drone continuaria subindo indefinidamente 
caso a entrada inserida fosse forte o bastanta para vencer a resistência aplicada pelo próprio peso do aparelho. Dessa 
forma, um drone é considerado um sistema BIBO instável.

Já para Lyapunov, que é um critério mais abrangente que o anterior, busca-se verificar a convergência das respostas 
do sistema em direção ao equilíbrio. Quando estamos tratando de sistemas lineares, uma forma de verificar essa 
convergência é por meio dos autovalores da matriz $\mathbf{A}$ (definida anteriormente) de modo que, caso eles possuam 
partes reais negativas, o sistema é considerado assintoticamente estável segundo Lyapunov. Assim, os autovalores 
da matriz dada são:

\begin{equation*}
    \left[\begin{array}{cccccccccccc}
        0 & 0 & -0.534 & -0.534 & 0 & 0 & 0 & 0 & -0.534 & 0 & 0 & 0
    \end{array}\right].
\end{equation*}

Observando os valores acima, nota-se que o sistema não pode ser considerado assintoticamente estável segundo Lyapunov 
pois nem todas as partes reais de seus autovalores são negativas, havendo vários deles nulos.

\subsection{Controlabilidade}
A controlabilidade está associada à capacidade de um sistema ser controlado de forma eficaz. Em outras palavras, ela 
analisa a capacidade das entradas disponíveis de levar o sistema, em um espaço de tempo finito, de qualquer um de seus 
estados, considerado como inicial, até qualquer outro estado, considerado como final.

Na prática, para sistemas lineares em formato de espaço de estados (ver seção anterior) essa análise é feita a 
partir da matriz de controlabilidade (mostrada abaixo), em que, para que o sistema 
seja considerado controlável, o posto dessa matriz deve ser igual à quantidade $n$ de estados do sistema (neste trabalho, 
$n=12$).

\begin{equation*}
    \mathbf{C_M} = \begin{bmatrix}
    B & AB & A^2B & \cdots & A^{n-1}B
    \end{bmatrix}
\end{equation*}

Realizando os devidos cálculos com as matrizes $\mathbf{A}$ e $\mathbf{B}$ dadas anteriormente, obtém-se um posto 
completo e, portanto, conclui-se que o sistema é controlável.

\subsection{Observabilidade}
\lipsum[1]

\subsection{Acoplamento}
\lipsum[1]