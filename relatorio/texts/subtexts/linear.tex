\section{Linearização \label{sec:linear}}
Observando as equações diferenciais apresentadas na seção anterior, é fácil notar que a dinâmica do drone possui 
bastante não-linearidade na maioria de suas relações. Contudo, uma vasta gama de técnicas de controle se aplicam 
somente, ou mais eficazmente, em sistemas lineares. Tendo isso em mente, decidiu-se fazer uma linearização desse 
sistema em torno de um ponto de operação — que, nesse caso, corresponde ao instante em que a força gerada pelos 
rotores equilibra exatamente o peso do equipamento — a fim de ser possível a utilização de técnicas de controle 
como o LQR, bem como de algumas técnicas de análise do sistema (ver próxima seção).

A linearização em si é um processo que utiliza a expansão em série de Taylor e, ao desprezar os termos de ordem 
superior, obtém-se um modelo aproximado do tipo $\dot{x}' = \mathbf{A} x' + \mathbf{B} u'$ e $y' = \mathbf{C} x' 
+ \mathbf{D} u'$, no qual:

\begin{itemize}
    \item $x'$ e $u'$ representam pequenas variações em torno do ponto de operação escolhido;
    \item $\mathbf{A}$ representa as relações entre as variáveis de estado;
    \item $\mathbf{B}$ representa como as entradas influenciam as variáveis de estado;
    \item $\mathbf{C}$ representa como os estados se combinam para formar as saídas do sistema;
    \item $\mathbf{D}$ representa a influência direta das entradas nas saídas.
\end{itemize}

Assim, para o caso do sistema do quadcóptero apresentado, 
as matrizes linearizadas são as seguintes:

\[
\mathbf{A} =
\left[\begin{array}{cccccccccccc}
0 & 0 & 0 & 0 & 0 & 0 & 1 & 0 & 0 & 0 & 0 & 0 \\
0 & 0 & 0 & 0 & 0 & 0 & 0 & 1 & 0 & 0 & 0 & 0 \\
0 & 0 & 0 & 0 & 0 & 0 & 0 & 0 & 1 & 0 & 0 & 0 \\
0 & 0 & 0 & 0 & 0 & 0 & 0 & 0 & 0 & 1 & 0 & 0 \\
0 & 0 & 0 & 0 & 0 & 0 & 0 & 0 & 0 & 0 & 1 & 0 \\
0 & 0 & 0 & 0 & 0 & 0 & 0 & 0 & 0 & 0 & 0 & 1 \\
0 & 0 & 0 & 0 & 9.81 & 0 & -0.534 & 0 & 0 & 0 & 0 & 0 \\
0 & 0 & 0 & -9.81 & 0 & 0 & 0 & -0.534 & 0 & 0 & 0 & 0 \\
0 & 0 & 0 & 0 & 0 & 0 & 0 & 0 & -0.534 & 0 & 0 & 0 \\
0 & 0 & 0 & 0 & 0 & 0 & 0 & 0 & 0 & 0 & 0 & 0 \\
0 & 0 & 0 & 0 & 0 & 0 & 0 & 0 & 0 & 0 & 0 & 0 \\
0 & 0 & 0 & 0 & 0 & 0 & 0 & 0 & 0 & 0 & 0 & 0
\end{array}\right],
\label{mat:A_lin}
\]

\vspace{+0.3cm}

\[
\mathbf{B} = 
\left[
\begin{array}{cccccccccccc}
0 & 0 & 0 & 0\\
0 & 0 & 0 & 0\\
0 & 0 & 0 & 0\\
0 & 0 & 0 & 0\\
0 & 0 & 0 & 0\\
0 & 0 & 0 & 0\\
0 & 0 & 0 & 0\\
0 & 0 & 0 & 0\\
6.37e-6 & 6.37e-6 & 6.37e-6 & 6.37e-6\\
0 & -1.38e-4 & 0 & 1.38e-4\\
-1.38e-4 & 0 & 1.38e-4 & 0\\
-1.30e-5 & 1.30e-5 & -1.30e-5 & 1.30e-5
\end{array}\right],
\]

\vspace{+0.3cm}

\[
\mathbf{C} = I_{12},
% \left[
% \begin{array}{cccccccccccc}
% 1 & 0 & 0 & 0 & 0 & 0 & 0 & 0 & 0 & 0 & 0 & 0\\
% 0 & 1 & 0 & 0 & 0 & 0 & 0 & 0 & 0 & 0 & 0 & 0\\
% 0 & 0 & 1 & 0 & 0 & 0 & 0 & 0 & 0 & 0 & 0 & 0\\
% 0 & 0 & 0 & 1 & 0 & 0 & 0 & 0 & 0 & 0 & 0 & 0\\
% 0 & 0 & 0 & 0 & 1 & 0 & 0 & 0 & 0 & 0 & 0 & 0\\
% 0 & 0 & 0 & 0 & 0 & 1 & 0 & 0 & 0 & 0 & 0 & 0\\
% 0 & 0 & 0 & 0 & 0 & 0 & 1 & 0 & 0 & 0 & 0 & 0\\
% 0 & 0 & 0 & 0 & 0 & 0 & 0 & 1 & 0 & 0 & 0 & 0\\
% 0 & 0 & 0 & 0 & 0 & 0 & 0 & 0 & 1 & 0 & 0 & 0\\
% 0 & 0 & 0 & 0 & 0 & 0 & 0 & 0 & 0 & 1 & 0 & 0\\
% 0 & 0 & 0 & 0 & 0 & 0 & 0 & 0 & 0 & 0 & 1 & 0\\
% 0 & 0 & 0 & 0 & 0 & 0 & 0 & 0 & 0 & 0 & 0 & 1\\
% \end{array}\right],
\]
\vspace{-0.5cm}
\[
\mathbf{D} = 0.
\]