\section{Controle Preditivo Baseado em Modelo Não-Linear (NMPC)}
\subsection{Conceito}
O NMPC é uma técnica de controle ótimo que utiliza um modelo matemático não linear do sistema para 
prever seu comportamento futuro e calcular, em tempo real, as melhores ações de controle. Nesse contexto, 
uma de suas características relevantes é a sua capacidade de considerar simultaneamente o desempenho 
desejado e os limites operacionais de modo que ele possui uma boa eficácia quando confrontado com 
sistemas reais complexos. Além disso, essa técnica opera sobre dois horizontes: o de predição ($\mathrm{P}$), que define 
quantos passos à frente o controlador prevê, e o de controle ($\mathrm{M}$), que determina quantas ações de 
controle serão planejadas para atingir a referência desejada.

A cada instante, o controlador resolve um problema de otimização que busca minimizar uma função de custo, normalmente composta 
pelo erro entre as saídas previstas e as desejadas, além do esforço de controle. Neste trabalho, a 
função de custo é a seguinte:
\begin{equation*}
    C = \mathrm{\int_{t}^{t+P} x^{T}_{e}Qx_e+u^{T}_{e}Ru_e \,d\tau},
\end{equation*}
\noindent em que $\mathrm{P}$, como dito anteriormente, é o horizonte de predição, $\mathrm{Q}$ 
é a matriz positiva que pondera os desvios nas variáveis de estado e $\mathrm{R}$ é a matriz 
positiva que pondera os desvios nos esforços de controle.

Apenas a primeira ação calculada é aplicada ao sistema, e o processo é repetido continuamente 
com base no novo estado medido — um procedimento conhecido como horizonte deslizante 
\textit{(receding horizon)}. Dessa forma, o NMPC ajusta suas decisões a cada iteração, garantindo 
um controle adaptativo e robusto frente a não linearidades e perturbações do sistema.

\subsection{Implementação}
Para implementar o NMPC foi utilizado o \textit{Otimizador de Ponto Interior (IPOPT)}, cuja aplicação 
é voltada para problemas de otimização não-linear em grande escala, em conjunto com o \textit{CasADI} 
(ferramenta de modelagem simbólica), ambos disponíveis em Python. Além disso, a Tabela 
\ref{tab:nmpc_params} explicita os parâmetros utilizados na implementação do NMPC.

\vspace{-0.5cm}
\begin{center}
    \begin{longtable}{|c|c|c|} % <-- note o uso de p{largura}
        \caption{Condições Aplicadas no NMPC} \vspace{-0.4cm}
        \label{tab:nmpc_params}
        \hline
        \emph{Condição} & \emph{Tipo} & \emph{Valor} \\
        \hline
        \endfirsthead
        \endhead
        \endfoot
        \caption*{\footnotesize{Fonte: Autora.}} \vspace{-0.55cm}
        \endlastfoot

        Horizonte de Predição ($P$)                & Parâmetro                & 30 passos               \\ \hline
        Horizonte de Controle ($M$)                & Parâmetro                & 30 passos               \\ \hline
        Set Point                                  & Parâmetro                & Circuito pré-definido   \\ \hline
        Erro de posição em $\mathtt{X, Y, Z}$      & Penalização              & -                       \\ \hline
        Excesso de velocidade linear               & Penalização              & -                       \\ \hline
        Excesso de velocidade angular              & Penalização              & -                       \\ \hline
        Dinâmica do modelo                         & Restrição                & -                       \\ \hline
        Ângulos $\phi$ e $\theta$  & Restrição                & $\leq 5^{\circ}$               \\ \hline
    \end{longtable}
\end{center}
\vspace{-1.5cm}

% \begin{table}[h!]
%     \centering
%     \caption{Condições Aplicadas no NMPC}
%     \label{tab:nmpc_params}
%     \begin{tabular}{|c|c|c|}
%         \hline
%         \emph{Condição}                            & \emph{Tipo}              & \emph{Valor}          \\ \hline
%         Horizonte de Predição ($P$)                & Parâmetro                & 30 passos               \\ \hline
%         Horizonte de Controle ($M$)                & Parâmetro                & 30 passos               \\ \hline
%         Set Point                                  & Parâmetro                & Circuito pré-definido   \\ \hline
%         Erro de posição em $\mathtt{X, Y, Z}$      & Penalização              & -                       \\ \hline
%         Excesso de velocidade linear               & Penalização              & -                       \\ \hline
%         Excesso de velocidade angular              & Penalização              & -                       \\ \hline
%         Dinâmica do modelo                         & Restrição                & -                       \\ \hline
%         Ângulos $\phi$ e $\theta$  & Restrição                & $\leq 5^{\circ}$               \\ \hline
%     \end{tabular}
% \end{table}

% \pagebreak

% \subsection{Resultados}